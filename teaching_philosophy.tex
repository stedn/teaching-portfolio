% LaTeX

\documentclass[12pt]{amsart} \usepackage{amssymb}

%\textwidth = 460pt 
%\textheight = 9in 
%\hoffset=-54pt \voffset=-40pt

% SIDE MARGINS:
\oddsidemargin 0in \evensidemargin 0in

% VERTICAL SPACING:
%\topmargin -.15in
\topmargin -.4in
\headheight 0in \headsep 0.0in
%\footheight 0.5in
\footskip 0.5in

\pagestyle{plain}
%\pagenumbering{}

% DIMENSION OF TEXT:
\textheight 10in \textwidth 6.5in
%

%\textwidth = 470pt
%\textheight = 700pt
%%\topmargin = 0pt
%%\oddsidemargin = 0pt
%\hoffset = -60pt
%\voffset = -50pt

%\input epsf \def\epsfsize#1#2{0.4#1\relax} \def\nl{\hfil\break}

%\renewcommand{\baselinestretch}{1.2}
%\def\Indent{\hskip .2in}


\title[]{C3 Teaching Philosophy}

\author[]{William McFadden}

\begin{document}
\maketitle
\thispagestyle{empty}

A scientific background provides a person with a useful practice of approaching and solving complex problems. Therefore, above all, I believe that teaching STEM disciplines at the college level should focus on equipping students to utilize a body of existing knowledge to develop and explain solutions. By the time I meet undergraduate students, most of them have only a few brief years before they will leave school and begin to act as collaborative and independent problem solvers. Ultimately, I feel that the value of any teaching I do will be judged mostly by what good it does for them once they are engaged in the practice of solving problems outside of school. 

In my experiences lecturing or leading discussions in physics, geophysics, computer science, and biology, I've realized that, regardless of discipline, the majority of the value added to the student comes from their becoming able to find, discuss, and convincingly utilize information to solve problems. I've also realized that my directly stating this knowledge in a large room does little to solidify this ability in them. As a result, my course structure, regardless of specific technical subject matter, has evolved to more closely resemble a guided practice in taking in information, discussing it, and then applying it to an independent problem. 

Although this is a quite fundamental notion that many educators would agree with, it can be hard to keep this in mind when building the day-to-day class plan.  To help me remember to focus on these three key parts that go into using knowledge to solve problems, I developed a teaching structure which I've nicknamed with the mnemonic C3 for the 3 core objectives that I center my teaching around. In this "C3" scheme, I have students read \textbf{core} material, \textbf{collaborate} with peers to shape their understanding, and \textbf{convey} solutions to problems. My only purpose as an educator is to set up a situation where my students are repeatedly guided through these three fundamental processes: observing, discussing, and presenting a solution.

In practice, this structure can take on different forms.  In its earliest conception it developed out of the way I prepared for my TAships in non-major science courses, hoping at least to equip my students with the useful skill of scientific problem solving.  In each discussion, I would briefly introduce the core concepts, then the students would discuss and work together, and finally they would complete assignments.  Since I was also frequently a grader, I guided them in their assignments to always focus on the exposition of their reasoning as more important than the traditional "showing your work." 

In other cases, I've been able to be even more explicit with the C3 structure. Although it might seem a little pedantic to explain your pedagogical style to students, I now prefer to firmly state the C3 structure and use it as the guidepost in constructing assignments as well.  In practice, this is extremely beneficial.  I've recently converted a purely technical, two-week workshop into a fundamental lesson on scientific communication simply by sating that there must be a phase of collaboration and a phase of conveyance.  In my most recently designed course, I've set up each lesson to directly follow this process. AND HERE I EXPLAIN HOW THAT ONE WENT fingers crossed

I've experimented with this structure outside of undergrad teaching as well.  In my after-school robotics lessons, I try to use this philosophy in the slightly more mayhem-infused middle-schooler classroom. Each class began with a cursory explanation of the fundamentals behind the technical challenge, then proceeded to group activities, and then ended with the completion of an individual project.  


\end{document}
