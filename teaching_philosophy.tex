% LaTeX

\documentclass[12pt]{amsart} \usepackage{amssymb}

%\textwidth = 460pt 
%\textheight = 9in 
%\hoffset=-54pt \voffset=-40pt

% SIDE MARGINS:
\oddsidemargin 0in \evensidemargin 0in

% VERTICAL SPACING:
%\topmargin -.15in
\topmargin -.8in
\headheight 0in \headsep 0.0in
%\footheight 0.5in
\footskip 0.5in

\pagestyle{plain}
%\pagenumbering{}

% DIMENSION OF TEXT:
\textheight 10in \textwidth 6.5in
%

%\textwidth = 470pt
%\textheight = 700pt
%%\topmargin = 0pt
%%\oddsidemargin = 0pt
%\hoffset = -60pt
%\voffset = -50pt

%\input epsf \def\epsfsize#1#2{0.4#1\relax} \def\nl{\hfil\break}

%\renewcommand{\baselinestretch}{1.2}
%\def\Indent{\hskip .2in}


\title[]{Statement of Teaching Philosophy}

\author[]{William McFadden}

\begin{document}
\maketitle
\thispagestyle{empty}

The objective of my teaching is for every student to learn the not just the tools but also the {\em experience} of solving hard problems.  By the time I meet undergraduate students, most of them have only a few brief years before they will leave school and begin to act as collaborative and independent problem solvers. Ultimately, the value of any teaching I do will be judged most by what good it does for them once they are engaged in a career outside of school. While other disciplines can have other goals, I believe that teaching STEM at the college level can do the most good for students by focusing on equipping them to approach problems, collaborate with others, and justify solutions. 

My teaching practices have grown out of a systematic attempt to simulate the experience of tackling a hard problem using a scientific approach. I've worked to build a functional yet general class structure to guide students through the process and motivate their learning through focused assessments.  With my teaching experience to date, I've experimented with this structure, observing its benefits and, in some cases, its limitations.

\section*{The 3 Cs}
In my experiences lecturing or leading discussions in physics, geophysics, computer science, and biology, I've realized that, regardless of discipline, the majority of the value added to the student comes from their becoming able to find, discuss, and convincingly utilize information to solve problems. As a result, my course structure, regardless of specific technical subject matter, has evolved to become a guided practice in three specific practices: 1) taking in information from a variety of sources, 2) discussing problems with peers, and 3) translating information and discussions to independent problem solving.

Although these three practices are nothing revolutionary and most scientists would probably agree with their importance, it can be hard to keep this in mind when building the day-to-day class plan.  To help me remember to focus on these three key parts that go into using knowledge to solve problems, I developed a teaching structure which I've nicknamed with the 3 Cs mnemonic for the 3 core objectives that I center my teaching around. In this "C3" scheme, I have students \textbf{collect} informational material, \textbf{collaborate} with peers to shape understanding, and \textbf{convey} solutions to problems. My only purpose as an educator is to set up a situation where my students are repeatedly guided through these three fundamental processes: observing, discussing, and presenting a solution.

By clearly structuring my courses around these three activities, it's possible for students to explicitly focus on these goals at the time when they are most appropriate.  When first introduced to a topic, students have to collect information to understand what is already known about the context of this problem. They learn that information sources can take many forms: a lecture, a youtube video, a textbook, an online tutorial or their own memories.  Once they have internalized some information, they are able to collaborate with their peers. Through this experience, they come to understand collaboration as necessary to check their personal understanding and to gather more information outside their own experience.  Finally, when they believe they can solve their problem, they learn that conveying their reasoning and methods is often the most important stage in delivering the solution.  

This structure has consistently given me a focused method to go about evaluating my teaching practices and my student's learning progress.  In the future, I hope that this structure can be refined further to allow me to systematically improve my teaching practices.

\section*{Experiences}
In practice, this structure can take on different forms depending on the subject matter and the background and goals of the student.
   
\subsection*{Discussion Sections}
My thinking about science education as broadly skill-based has developed out of the way I prepared for my teaching in non-major science courses. I was confronted with students who truly didn't feel that they needed to know the material, and I recognized that it was quite true that most of the content that was being emphasized was actually quite secondary to the needs of these students. Seeing it from their perspective, I realized that the best I could do for my students was at least to equip them with the useful skill of scientific problem solving.  When helping one of my business students in my Chemistry of the Atmosphere course, I approached one of our early lessons on dimensional analysis as if it were prep for an interview at a consulting firm.  It really seemed like this helped him to motivate his learning, and by the end of the session he was spontaneously collaborating with another student to develop his skills.  If I could have given him extra points for his collaborative spirit, I would have, but it turned out that I didn't need to---this effort appeared to translate into really great scores on the rest of his assignments anyway.  For another student who was struggling with computer labs in my Global Warming discussion section, I spent some time to explain how important it was to clearly convey his ideas in his assignments.  From that suggestion, he came up with the idea to type his homework, which made his reasoning much more transparent (to him and me) and helped raise his grades too.  While I've had successes with many students by emphasizing these skills informally, for other students, there wasn't enough immediately apparent benefit to motivate them to take these kinds of actions.  This is one reason that I feel I must directly incorporate more explicit emphasis on these competencies in my assessment strategies once I'm building my own courses.

\subsection*{After-school Clubs}
I've experimented with this structure outside of undergrad teaching as well in the design of my outreach program's full-year middle-schooler robotics club.   The curriculum structure begins with a 10 club sessions collecting information and working on small scale group activities.  During this time, the students become acquainted both with state-of-the-art tools in electronics and programming as well as the vast array of DIY troubleshooting help available on the web.  They also build relationships with their classmates in small-scale projects, like building a laser trip-wire for a DIY home security system.  After learning some basics, the curriculum moves to building standalone robots capable of remote control from a laptop for robot v. robot soccer. The students work together in teams to improve their designs and outperform other clubs. Finally, during summer camp, the students make individual home sensor kits and build websites to convey their technical challenges and scientific findings to the world.   It's important to keep in mind that this entire curriculum is a {\em non-mandatory} extracurricular activity for hyper-active preteens---every lesson and project must be intrinsically motivating outside the structure of academic credit.  Fortunately, making cool things work is super-freaking-cool and impressive so many students are able to find enjoyment and pride in working through the solution of genuinely tricky problems.  Nevertheless, every year we do have students who simply will not behave orderly and participate productively when we try to engage them this way.  I don't have a perfect answer for how to treat these students, but for our current approach, if a student in our class decides that they would prefer to use their time and access to computers to play internet video games, we let them make their own choices after reminding them about the goals of the class. Through two years running this program, I've now seen the look on those students' faces when they realize near the end of the year that their classmates have created something awesome in the same place as them, with the same time that they had spent on games.  I don't yet know what impact that experience will have on those students, but I believe that the feeling of such missed opportunity may be a valuable but neglected learning experience.  Most importantly, through this experience I've become keenly aware of the sometimes surprising lack of internal motivation in young adults (including myself), so I feel it's important not to forget this lesson when developing STEM curriculum for college students as well.

\subsection*{Graduate Workshops}
In graduate workshops that I've put on myself, I've been able to be even more explicit with the C3 structure. Although it might seem a little pedantic to explain my pedagogical style to students, I now prefer to firmly state the C3 structure and use it as the guidepost in constructing assignments as well.  In practice, this has proved to be extremely beneficial for keeping the students focused on the course's core goals since my method of assessment is directly structured around the C3 scheme to keep my students aware of my goals.  For my most recent 2-week graduate workshop, I explicitly used my 3 aims to motivate their main assignment, a presentation. Although I wasn't giving this lesson for a formal grade, I assessed their project using equal parts basic content knowledge, collaborative work, and presentation quality. The students understood the goals and their motivation, and this clearly motivated them to far exceed my expectations on a non-credit assignment.  

\section*{Games and Economies in the Classroom}
As the majority of the work I've done with both middle-schoolers and college students has been in situations where I am not able to award academic grades, I've had to be creative in keeping students interested in their learning.  In particular, it can be quite difficult to motivate the small, daily tasks, that accumulate into larger learning goals.  This has led me to experiment with assessment structures that rely strongly on games and economics to couch certain monotonous tasks.  I've found that my students are motivated by competition far beyond my expectations.  In my future teaching endeavors, I'm highly interested in exploring the value of competition, games, and markets to motivate learning.  I think this is a rich landscape to explore problem solving based education, and I look forward to further opportunities to utilize the C3 structure on digital scales.



\end{document}
