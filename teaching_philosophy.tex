% LaTeX

\documentclass[12pt]{amsart} \usepackage{amssymb}

%\textwidth = 460pt 
%\textheight = 9in 
%\hoffset=-54pt \voffset=-40pt

% SIDE MARGINS:
\oddsidemargin 0in \evensidemargin 0in

% VERTICAL SPACING:
%\topmargin -.15in
\topmargin -.4in
\headheight 0in \headsep 0.0in
%\footheight 0.5in
\footskip 0.5in

\pagestyle{plain}
%\pagenumbering{}

% DIMENSION OF TEXT:
\textheight 10in \textwidth 6.5in
%

%\textwidth = 470pt
%\textheight = 700pt
%%\topmargin = 0pt
%%\oddsidemargin = 0pt
%\hoffset = -60pt
%\voffset = -50pt

%\input epsf \def\epsfsize#1#2{0.4#1\relax} \def\nl{\hfil\break}

%\renewcommand{\baselinestretch}{1.2}
%\def\Indent{\hskip .2in}


\title[]{Statement of Teaching Philosophy}

\author[]{William McFadden}

\begin{document}
\maketitle
\thispagestyle{empty}

The objective of my teaching is for every student to learn the tools and experience of problem solving.  By the time I meet undergraduate students, most of them have only a few brief years before they will leave school and begin to act as collaborative and independent problem solvers. Ultimately, the value of any teaching I do will be judged most by what good it does for them once they are engaged in a career outside of school. While other disciplines can have other goals, I believe that teaching STEM at the college level can do the most good for students by focusing on equipping them to approach problems, collaborate with others, and justify solutions. 

My teaching practices have grown out of a systematic attempt to simulate the experience of tackling a hard problem using a scientific approach. I've worked to build a functional yet general class structure to guide students through the process and motivate their learning through focused assessments.  With my teaching experience to date, I've experimented with this structure, observing its benefits and, in some cases, its limitations.

\section*{C3 Structure}
In my experiences lecturing or leading discussions in physics, geophysics, computer science, and biology, I've realized that, regardless of discipline, the majority of the value added to the student comes from their becoming able to find, discuss, and convincingly utilize information to solve problems. As a result, my course structure, regardless of specific technical subject matter, has evolved to become a guided practice in three specific practices: 1) taking in information from a variety of sources, 2) discussing problems with peers, and 3) translating information and discussions to independent problem solving.

Although these three practices are nothing revolutionary and most scientists would probably agree with their importance, it can be hard to keep this in mind when building the day-to-day class plan.  To help me remember to focus on these three key parts that go into using knowledge to solve problems, I developed a teaching structure which I've nicknamed with the mnemonic C3 for the 3 core objectives that I center my teaching around. In this "C3" scheme, I have students read \textbf{collect} informational material, \textbf{collaborate} with peers to shape understanding, and \textbf{convey} solutions to problems. My only purpose as an educator is to set up a situation where my students are repeatedly guided through these three fundamental processes: observing, discussing, and presenting a solution.

By clearly structuring my courses around these three activities, it's possible for students to explicitly focus on these goals at the time when they are most appropriate.  When first introduced to a topic, students have to collect information to understand what is already known about the context of this problem. They learn that information sources can take many forms: a lecture, a youtube video, a textbook, or their own memories.  When they have internalized some information, they are able to collaborate with their peers. Through this experience, they come to understand collaboration as necessary to check their personal understanding and to gather more information outside their own experience.  Finally, when they believe they can solve their problem, they learn that conveying their reasoning and methods is often the most important stage in delivering the solution.  

This structure has consistently given me a focused method to go about evaluating my teaching practices and my student's learning progress.  In the future, I hope that this structure can be refined further to allow me to systematically improve my teaching practices.

\section*{Experiences}
In practice, this structure can take on different forms depending on the subject matter and the background and goals of the student. In my discussion TAships, I can use the time in class to show students the importance of clearly conveying the knowledge they've acquired in lectures.   In my middle-school robotics clubs C3 takes the form of collaborative group projects and student explanations of their work.  Finally, in the courses and workshops that I teach myself, I implement a complete assessment structure to produce a fledgling "education economy".   
\subsection*{Discussion Sections}
My thinking about science education as broadly skill-based has developed out of the way I prepared for my teaching in non-major science courses. I was confronted with students who truly didn't feel that they needed to know the material, and I recognized that it was quite true that most of the content that was being emphasized was actually quite secondary to the needs of these students. Seeing it from their perspective, I realized that the best I could do for my students was at least to equip them with the useful skill of scientific problem solving.  In each discussion, I would briefly introduce the core concepts, then the students would discuss and work together, and finally they would complete assignments.  Over time, I formalized my process to try to balance my emphasis between understanding a set of facts and conveying the problem solution.

\subsection*{After-school Clubs}
I've experimented with this structure outside of undergrad teaching as well.  In my after-school robotics lessons, I try to use this philosophy in the slightly more mayhem-infused middle-schooler classroom. Each class began with a cursory explanation of the fundamentals behind the technical challenge, then proceeded to group activities, and then ended with the completion of an individual project.  


\subsection*{Graduate Workshops}
In graduate workshops that I've put on myself, I've been able to be even more explicit with the C3 structure. Although it might seem a little pedantic to explain my pedagogical style to students, I now prefer to firmly state the C3 structure and use it as the guidepost in constructing assignments as well.  In practice, this has proved to be extremely beneficial for keeping the students focused on the course's core goals.  My method of assessment is directly structured around the "C3" scheme to keep my students aware of my goals.  For my most recent 2-week graduate workshop, I explicitly used my 3 aims to motivate their main assignment, a presentation. Although I wasn't giving this lesson for a formal grade, I assessed their project using equal parts basic content knowledge, collaborative work, and presentation quality. The students understood the goals and their motivation, and this clearly motivated them to far exceed my expectations on a non-credit assignment.  

\section*{Games and Economies in the Classroom}
As the majority of the work I've done with both middle-schoolers and college students has been in situations where I am not able to award academic grades, I've had to be creative in keeping students interested in their learning.  In particular, it can be quite difficult to motivate the small, daily tasks, that accumulate into larger learning goals.  This has led me to experiment with assessment structures that rely strongly on games and economics to couch certain monotonous tasks.  I've found that my students are motivated by competition far beyond my expectations.  In my future teaching endeavors, I'm highly interested in exploring the value of competition, games, and markets to motivate learning.  I think this is a rich landscape to explore problem solving based education, and I look forward to further opportunities to utilize the C3 structure on digital scales.



\end{document}
